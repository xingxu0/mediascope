
%\vspace*{-3ex}
%

Motivated by an availability gap for visual media, where images and
videos are uploaded from mobile devices well after they are generated,
we explore the \emph{selective,
  timely retrieval} of media content from a collection of mobile
devices.
%
We envision this capability being driven by \emph{similarity-based
  queries} posed to a cloud search front-end, which in turn
dynamically retrieves media objects from mobile devices that best
match the respective queries within a given time limit.
%
Building upon a crowd-sensing framework, we have designed and
implemented a system called \mscope that provides this capability.
%
\mscope is an extensible framework that supports nearest-neighbor and
other geometric que-ries on the feature space (e.g., clusters,
spanners), and contains novel retrieval algorithms that attempt to
maximize the retrieval of relevant information.
%
From experiments on a prototype, \mscope is shown to achieve
near-optimal query completeness and low to moderate overhead on mobile
devices.

% %
% In this paper, we design and implement $ProjectName$, a novel
% on-demanding media search system over mobile phones that requires no
% in-advance media objects uploading and supports different kinds of
% queries.
% %
% With tremendously increasing computational capability of mobile
% phones, $ProjectName$ designed efficient algorithm to extract a small
% but representative signature of media objects which can support
% different kinds of queries.
% %
% Bandwidth resource is scarce for the mobile phones especially when
% there are concurrent queries, to efficiently allocate such limited
% resources, $ProjectName$ proposed $\bf{credit-based}$ scheme to
% provide good results to all the queries.
% %
% In $ProjectName$ prototype, we implemented five different media
% queries, and $ProjectName$ provides high level abstraction and thus
% programmer to design their own query.
% %

%
