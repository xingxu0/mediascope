% Useful macros for paper writing

\definecolor{brown}{cmyk}{0,0.81,1,0.60}
\definecolor{magenta}{rgb}{0.4,0.7,0}
\definecolor{gray}{rgb}{0.5,0.5,0.5}
\definecolor{red}{rgb}{1,0,0}
\definecolor{green}{rgb}{0.5,0,0.5}
\definecolor{blue}{rgb}{0,0,1}
\newcommand{\etc}{\emph{etc.}\xspace}
\newcommand{\ie}{\emph{i.e.,}\xspace}
\newcommand{\eg}{\emph{e.g.,}\xspace}
\newcommand{\etal}{\emph{et al.}\xspace}
\newcommand{\vlc}{vlc\xspace}
\newcommand{\SNOOZE}{Snooze\xspace}
\newcommand{\snz}{Snooze\xspace}
\newcommand{\POLICE}{Police\xspace}
\newcommand{\police}{Police\xspace}
\newcommand{\Police}{Police\xspace}
\newcommand{\PSTA}{Police Station\xspace}
\newcommand{\RTH}{\textbf{A}\xspace}
\newcommand{\ISI}{\textbf{B}\xspace}
%\newcommand{\reducefiguretopvmargin}{\vspace*{0ex}}
%\newcommand{\reducefigurebottomvmargin}{\vspace*{0.0ex}}
\newcommand{\reducefigurebottomvmarginsmall}{\vspace{-3mm}}
\newcommand{\reducefigurebottomvmargin}{\vspace{-4mm}}
\newcommand{\reducefigurecaptionmargin}{\vspace{-0mm}}
\newcommand{\reduceparavmargin}{\vspace*{0.0ex}}
%\newcommand{\reduceparavmargin}{\vspace*{-2.0ex}}
\newcommand{\reducesectionvmargin}{\vspace*{0.0ex}}
%\newcommand{\reducesectionvmargin}{\vspace*{-1.0ex}}
\newcommand{\reducevmarginsmall}{\vspace*{-.5ex}}
\newcommand{\reducevmarginone}{\vspace*{-1.0ex}}
\newcommand{\reducevmargintwo}{\vspace*{-2.0ex}}
\newcommand{\reducevmarginthree}{\vspace*{-3.0ex}}
\newcommand{\reducevmarginfour}{\vspace*{-4.0ex}}
\newcommand{\reducevmarginfive}{\vspace*{-5.0ex}}
\newcommand{\mypar}[1]{\vspace*{0.5ex}\noindent\textbf{#1}}
\newcommand{\mypari}[1]{\vspace*{0.5ex}\noindent\emph{#1}}

\newcommand{\spacesavecaption}[1]{\vspace*{0.0ex}\caption{#1}\vspace*{-3.5ex}}

\newcommand{\smallsection}[1]{\vspace*{1ex}\noindent\textbf{#1}}

% comment
\newcommand{\comment}[1]{{\color{gray}[\textsf{#1}]}}
\newcommand{\binliu}[1]{{\color{green}(KJ: #1)}}
\newcommand{\fei}[1]{{\color{red}(AS: #1)}}
\newcommand{\ramesh}[1]{{\color{blue}(RG: #1)}}
\newcommand{\yurong}[1]{{\color{brown}(SH: #1)}}
\newcommand{\camera}[1]{#1}

%\hyphenation{rate-allo-c-ation}

\newcommand{\equaref}[1]{Eq.~(\ref{eq:#1})}
\newcommand{\figref}[1]{\ref{fig:#1}}
\newcommand{\algref}[1]{Alg.~(\ref{alg:#1})}
\newcommand{\ruleref}[1]{\textsc{Rule}~\ref{rule:#1}}
\newtheorem{rul}{Rule}
\newcommand{\nonoverlapping}{\mbox{non-o}\-ver\-lap\-ping\xspace}
\newcommand{\interframe}{interframe\xspace}

\newcommand{\script}[1]{{{\cal{#1}}}}


\newtheorem{theorem}{Theorem}[section]
\newtheorem{lemma}[theorem]{Lemma}
\newtheorem{proposition}[theorem]{Proposition}
\newtheorem{corollary}[theorem]{Corollary}

% \newenvironment{definition}[1][Definition]{\begin{trivlist}
% \item[\hskip \labelsep {\bfseries #1}]}{\end{trivlist}}
% \newenvironment{example}[1][Example]{\begin{trivlist}
% \item[\hskip \labelsep {\bfseries #1}]}{\end{trivlist}}
% \newenvironment{remark}[1][Remark]{\begin{trivlist}
% \item[\hskip \labelsep {\bfseries #1}]}{\end{trivlist}}

\newcommand{\bm}[1]{\mbox{\boldmath{$#1$}}}
\newcommand{\ppcl}{Pickle\xspace}



%Figures

\newcommand{\scaleImage}[4]{
\begin{figure}[#1]
\centering
\includegraphics[width=#2\textwidth]{figs/#3}
\reducefigurecaptionmargin
\caption{#4\label{fig:#3}}
\reducefigurebottomvmarginsmall
\end{figure}
}

\newcommand{\scaleImageLabel}[5]{
\begin{figure}[#1]
\centering
\includegraphics[width=#2\textwidth]{figs/#3}
\reducefigurecaptionmargin
\caption{#4\label{fig:#5}}
\reducefigurebottomvmarginsmall
\end{figure}
}


\newcommand{\fullColumnFigs}[3]
{
\begin{figure*}[!tb]
  \centering
  {#1}
\reducefigurecaptionmargin
\caption{#2\label{fig:#3}}
\reducefigurebottomvmargin
\end{figure*}
}

\newcommand{\scaleTable}[4]{
\begin{table}[#1]
\centering
\includegraphics[width=#2\textwidth]{figs/#3}
\reducefigurecaptionmargin
\caption{#4\label{tbl:#3}}
\reducefigurebottomvmarginsmall
\end{table}
}

\newcommand{\oneColumnFigs}[3]
{
\begin{figure}[t]
  \centering
  {#1}
\reducefigurecaptionmargin
\caption{#2\label{fig:#3}}
\reducefigurebottomvmargin
\end{figure}
}

\newcommand{\subImage}[3]{% width, filename1, caption1, label1
    \hspace*{-2.0ex}
    \subfigure[#3]
    {
      \includegraphics[width=#1\textwidth]{figs/#2}
      \label{fig:#2}
    }
}

\newcommand{\subImagePadded}[5]{% figure_width, hpadding, filename1, caption1, label1
    \subfigure[#4]
    {
      \hspace{#2\textwidth}
      \includegraphics[width=#1\textwidth]{figs/#3}
      \hspace{#2\textwidth}
      \label{fig:#5}
    }
}

\newcommand{\subImageWithNoLable}[2]{% width, filename1, caption1, label1
      \includegraphics[width=#1\textwidth]{figs/#2}
}

%%% Local Variables: 
%%% mode: latex
%%% TeX-master: "paper"
%%% End: 
