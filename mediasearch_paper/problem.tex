
\section{Problem Formulation}
\label{sec:probl}

Next, we will give a detailed problem formulation. System consists of central query server and registered users who are willing to reply the query task assigned by the query server. Whenever comes a query, the central server filters the information with query, and then adopts scheduling algorithm to calculate uploading photos with the photo's meta data. Given the schedule, the central server begins to assign tasks for the phones, and phones response and upload the assigned photos. To make the system model clear, we separate it into 3 parts: Filtering, Network Model and Information Model

First, let's talk about the first step of our system: Filtering. We build "big table" based on the meta data. Objects works as keys, and each column associated with it represents a photos, which contains $(timestamp, location, user_id)$ and gets sorted by timestamp. When the query comes, we search corresponding object rows and refer to the satisfied time range from columns. Next, we further filter these photos with location range. After that, we cluster the qualified photos by user identities, which is the input for the scheduling scheme.

Second, Network Model. After Filtering, we get a bunch of  media files index clustered by user identities. We assume there are $N$ users that own these media files, for user $i$ owns $a_{i}$ files, and each file $j$ has a size of $s_{i,j}$. Each user $i$'s bandwidth is $b_{i}$ and upload files sequentially, besides, they are independent with each other, which means their wireless communication won't affect each other. So in the network side, the scheduling should consider the network conditions in user side, and requires participated user to upload the best subset of files within deadline $T$.

Third, Information Model. Information model is the kernel part for scheduling. Network  gives a constraint for our system,
 so the remaining problem is how to select photos from the qualified photos under the constraint. We define a new term, called ${similarity}$, to describe the relationship between any pair of photos. To make problem clear, we take two files: file $i$ and file $j$ as an example here, the similarity vector ${\overrightarrow{v}_{i,j}}$ is composed by the following elements: time difference ${t_{i,j}}$, location difference ${d_{i,j}}$, object difference ${o_{i,j}}$, color distribution difference ${c_{i,j}}$, trace direction ${tr_{i,j}}$, etc. Since files are owned by different users, then file owner is also an important part for similarity, and we add this information as a binary value in similarity. So now we normalize these elements and define similarity as the weighted sum of these elements. With normalization, the similarity vector can be represented as follows:

 \begin{equation}
\begin{array}{l}
 {\overrightarrow{v}_{ij}} =  \\
 \left( {\frac{{{t_{i,j}}}}{{\mathop {\max }\limits_{i,j} \left( {{t_{i,j}}} \right)}},\frac{{{d_{i,j}}}}{{\mathop {\max }\limits_{i,j} \left( {{d_{i,j}}} \right)}},\frac{{{o_{i,j}}}}{{\mathop {\max }\limits_{i,j} \left( {{o_{i,j}}} \right)}},\frac{{{c_{i,j}}}}{{\mathop {\max }\limits_{i,j} \left( {{c_{i,j}}} \right)}},\frac{{t{r_{i,j}}}}{{\mathop {\max }\limits_{i,j} \left( {t{r_{i,j}}} \right)}}} \right) \\
 \end{array}
 \end{equation}



To quantify the similarity value, we define a weighted sum function for similarity vector, $f(\overrightarrow{v}_{i,j})$, which is a numerical value.

\begin{equation}
\begin{array}{l}
 f\left( {{\overrightarrow{v}_{ij}}} \right) = {\alpha _1}\frac{{{t_{i,j}}}}{{\mathop {\max }\limits_{i,j} \left( {{t_{i,j}}} \right)}} + {\alpha _2}\frac{{{d_{i,j}}}}{{\mathop {\max }\limits_{i,j} \left( {{d_{i,j}}} \right)}} +  \\
 {\alpha _3}\frac{{{o_{i,j}}}}{{\mathop {\max }\limits_{i,j} \left( {{o_{i,j}}} \right)}} + {\alpha _4}\frac{{{c_{i,j}}}}{{\mathop {\max }\limits_{i,j} \left( {{c_{i,j}}} \right)}} + {\alpha _5}\frac{{t{r_{i,j}}}}{{\mathop {\max }\limits_{i,j} \left( {t{r_{i,j}}} \right)}} \\
 \end{array}
\end{equation}

in which, $\alpha_{1}+\alpha_{2}+\alpha_{3}+\alpha_{4}+\alpha_{5}=1$, and each $\alpha$'s value is assigned by the query's preference, e.g. if commander prefers location difference, then $\alpha_{1}$ gets more percentage.

Consider the simplest case, each file is of same size and one query scenario. After filtering, we get $N$ photos from $M$ users, each user $i$ owns $a_{i-1}$ photos, that is, $a_{0}+\ldots+a_{M-1}=N$. With the bandwidth and deadline constraint, users can only upload limited qualified files, more precisely, for user $i$, he can upload $s_{i}$ files under the constraint, in which, $s_{i}\leq a_{i}$ and $s_{0}+\ldots+s_{M-1}=K$. The problem now is to select a subset $K$ photos from these $N$ photos, such that:
\begin{equation}
\sum\limits_{i = 0}^{K - 1} {\sum\limits_{j = 0}^{K - 1} {\sum\limits_{{k_i} = 0}^{{s_i} - 1} {\sum\limits_{{k_j} = 0}^{{s_j} - 1} {f\left( {{{\vec v}_{{k_i}{k_j}}}} \right)} } } }
 \end{equation}

 get maximized, in which, $f(\overrightarrow{v}_{k_{i}k_{j}})=0$ if $k_{i}=k_{j}$ and $i=j$. 